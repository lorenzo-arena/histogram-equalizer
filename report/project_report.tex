\documentclass[10pt,twocolumn,letterpaper]{article}

\usepackage{cvpr}
\usepackage{times}
\usepackage{epsfig}
\usepackage{graphicx}
\usepackage{amsmath}
\usepackage{amssymb}

\usepackage{url}

% Include other packages here, before hyperref.

% If you comment hyperref and then uncomment it, you should delete
% egpaper.aux before re-running latex.  (Or just hit 'q' on the first latex
% run, let it finish, and you should be clear).
%\usepackage[pagebackref=true,breaklinks=true,letterpaper=true,colorlinks,bookmarks=false]{hyperref}

\cvprfinalcopy % *** Uncomment this line for the final submission

\def\cvprPaperID{****} % *** Enter the CVPR Paper ID here
\def\httilde{\mbox{\tt\raisebox{-.5ex}{\symbol{126}}}}

% Pages are numbered in submission mode, and unnumbered in camera-ready
\ifcvprfinal\pagestyle{empty}\fi
\begin{document}

%%%%%%%%% TITLE
\title{PC-2020/21 Histogram Equalization}

\author{Lorenzo Arena\\
{\tt\small lorenzo.arena@stud.unifi.it}
}

\maketitle
\thispagestyle{empty}

%%%%%%%%% ABSTRACT
\begin{abstract}
    This project was made as the final work for the “Parallel Programming”
    exam at the University of Florence. The objective was to create three
    version of a software to equalize the histogram of a given JPEG image.
    The first version had to be implemented as a sequential program, while
    the other two had to be parallel; for the parallel versions one has
    been implemented using OpenMP on top of the already implemented sequential
    code and the other is built using CUDA to take advantage of a GPU computing
    power. The projects has been developed in C on a Ubuntu machine; all
    tests were made on a Ryzen 7 1700 CPU and a NVIDIA Quadro P2000.
 \end{abstract}

 %%%%%%%%% BODY TEXT
\noindent\large\textbf{Future Distribution Permission}\\
\indent The author(s) of this report give permission for this document to be
        distributed to Unifi-affiliated students taking future courses.

\section{The algorithm}

%%%%%%%%%% TO WRITE
%% https://en.wikipedia.org/wiki/Histogram_equalization
Histogram equalization in image processing can be used to increase an image's
contrast. This is accomplished by spreading the most frequent intensities
values across the whole histogram. It works especially well on images which
presents foreground and background which are both dark or both bright.

\subsection{implementation}

We can start by considering a grayscale image of \(n\) pixels, in which \(n_i\)
is the number of occurrences of gray level \(i\). The probability of an occurrance
of a pixel of level \(i\) (with \(0 < i < L\)) in the image is:

\[ p_x(i) = \frac{n_i}{n} \]

The histogram is the distribution of the pixel levels in the range \([0, L - 1]\).

We can define the \emph{cumulative distribution function} as the cumulative sum
of all the probabilities lying in its domain:

\[ cdf_x(i) = \sum_{j=0}^{i} p_x(x = j) \]

To get a linearized \(cdf\), which will produce a flatten histogram, we must
normalize the is to the \([0, L - 1]\) range by using the following formula:

\[ cdfn_x(i) = \frac{cdf(i) - cdf_{min}}{(M * N) - cdf_{min}} \]

where \(M\) and \(N\) are the image's dimensions. The normalized \(cdf\) must
then be applied to the original histgram:

\[ h(v) = round(cdfn_x(i) * (L - 1)) \]

While this can be applied to grayscale images by using the pixel value, for
color images applying the equalization process to the R, G and B channels would
break the image's color balance since the relative distribution of the color
channels would be changed. Thus, the pixels must be converted to another color
space (like HSL) where the algorithm can be applied to the \emph{luminance}
channel without creating changes in the \emph{hue} or \emph{saturation}
channels.

\section{The sequential implementation}

%%%%%%%%%% TO WRITE

\section{The OpenMP implementation}

%%%%%%%%%% TO WRITE

\section{The CUDA implementation}

%%%%%%%%%% TO WRITE

\section{Results}

%%%%%%%%%% TO WRITE

{\small
\bibliographystyle{ieee}
\bibliography{egbib}
}

\end{document}